
\fancyhf{}
\fancyhead[c]{Chapter 8. Appendices}% <- added
\fancyfoot[R]{\thepage\ifodd\value{page}\else\hfill\fi}
%\fancyhead[L]{\ifodd\value{page}\relax\else\hfill\fi Ch \thechapter}
%\renewcommand\headrulewidth{0pt}% default ist .4pt
\renewcommand{\plainheadrulewidth}{.4pt}% default is 0pt

\section{Nederlandse samenvatting}
\tab Dit proefschrift bevat cross-sectionele en longitudinale onderzoeken die verbanden tussen voeding en gedrag onderzochten. Voor gedrag is er onder andere specifiek gekeken naar executieve functies en inhibitievermogen. Executieve functies zijn belangrijke processen die plaatsvinden in het brein om doelgerichte handelingen zo efficiënt mogelijk uit te voeren. Hierbij kan men bijvoorbeeld denken aan het bakken van een taart: wanneer de oven wordt aangezet voordat het beslag wordt gemaakt, dan is de oven al op temperatuur wanneer het beslag klaar is. Inhibitie, ofwel het vermogen om impulsen te beheersen, speelt een belangrijke rol in deze executieve functies.

De onderzochte verbanden in dit proefschrift reiken van het vroege leven tot de adolescentie. De darmbacterie-brein as, ofwel de communicatieroute tussen de darmbacteriën en het brein, is een belangrijk mechanisme dat de verwachte verbanden zou kunnen verklaren. Het eerste doel van dit proefschrift was om de relaties te onderzoeken tussen borstvoedingsfactoren en gedrag bij peuters. Het tweede doel was om de relaties te onderzoeken tussen darmbacteriën en gedrag bij peuters. Het derde doel was om de rol van de kwaliteit van de zorg van de moeder, meerdere keren gemeten tussen geboorte van het kind en de adolescentie, in het voedingsgedrag van adolescenten te onderzoeken.

In \textbf{hoofdstuk 2} is onderzocht of de lengte van de borstvoedingsperiode, ofwel borstvoedingsduur, het inhibitievermogen van een peuter voorspelt. Vervolgens is onderzocht of de dieetkwaliteit van een peuter een tussenrol speelt in deze voorspelling. In de eerste drie jaren na de bevalling, hebben moeders de borstvoedingsduur bijgehouden. Op driejarige leeftijd is het inhibitievermogen van het kind in kaart gebracht met behulp van vier verschillende gedragstaken. Ook zijn er vragenlijsten bij beide ouders afgenomen om het gedrag van het kind in kaart te brengen. De voedingsvragenlijst over de voedingsinname van de peuters is ingevuld door één van de ouders. Er is geen bewijs gevonden voor een verband tussen borstvoedingsduur en inhibitievermogen. Resultaten van voorgaand onderzoek op dit onderwerp liepen al uiteen, waarschijnlijk onder meer door het gebruik van verschillende meetmethoden. Net als voorgaand onderzoek wees ons onderzoek wel uit dat langere borstvoedingsduur een betere dieetkwaliteit van de peuters voorspelt. Het is echter niet duidelijk wat het mechanisme achter dit verband is. Hoe belangrijk moeders gezonde voeding vinden is mogelijk een belangrijke speler in dit verband.

In \textbf{hoofdstuk 3} is er onderzocht of humane melk oligosachariden (HMOs), de executieve functies en het inhibitievermogen van peuters voorspelt. HMOs zijn complexe suikers die aanwezig zijn in moedermelk. Moeders hebben hiervoor op twee, zes en 12 weken een kleine hoeveelheid melk verzameld. Van deze melk is de HMO-samenstelling geanalyseerd. Dezelfde meetmethoden benoemd in hoofdstuk 2 zijn gebruikt om executieve functies en inhibitie bij de peuters te meten. Resultaten lieten zien dat hoge niveaus van gefucosyleerde HMOs gerelateerd zijn aan betere executieve functies bij peuters. Er is geen bewijs gevonden voor een verband tussen gesialyleerde HMOs en executieve functies bij peuters. Onze resultaten wat betreft gefucosyleerde HMOs komen overeen met voorgaand soortgelijk onderzoek in dieren en mensen. Onze resultaten over gesialyleerde HMOs komen gedeeltelijk overeen met soortgelijke studies. Dit komt omdat er uiteenlopende resultaten zijn gevonden in voorgaand onderzoek. Deze verschillen zijn te verklaren door het gebruik van verschillende onderzoeksmethoden en de leeftijden waarop het gedrag van de kinderen is gemeten. Dit is het eerste onderzoek bij mensen dat HMOs heeft gemeten op drie tijdspunten binnen de eerste drie maanden. Replicatie van dit onderzoek is daarom belangrijk.

In \textbf{hoofdstuk 4} zijn de relaties tussen darmbacteriën en executieve functies (inclusief inhibitievermogen) van peuters onderzocht. Dezelfde meetmethoden benoemd in hoofdstuk 2 zijn gebruikt om de executieve functies en inhibitie te meten. Op twee, zes en 12 weken en op één en drie jaar, is de compositie van de darmbacteriën van het kind geanalyseerd. Er zijn verbanden gevonden tussen hogere niveaus van \textit{Streptococcus}, [\textit{Ruminococcus}] \textit{Torques} groep, \textit{Clostridium sensu stricto} 1, \textit{Intestinibacter}, en \textit{Halomonas} en verminderde executieve functies bij peuters. Een hoger niveau van \textit{Bacteroides}, \textit{Parabacteroides}, \textit{Ruminococcus} 2, en \textit{Blautia} en een hogere diversiteit aan verschillende soorten darmbacteriën bleken  gerelateerd aan betere executieve functies. Hogere niveaus van \textit{Bacteroides}, \textit{Ruminococcaceae} UCG-013 en \textit{Veillonella} voorspelden betere inhibitievermogen. Hogere niveaus van \textit{Subdoligranulum}, \textit{Lachnospiraceae} NK4A136, \textit{Anaerostipes}, \textit{Sutterella} en \textit{Coprococcus} 3 voorspelden verminderde inhibitievermogen in peuters. Resultaten van voorgaande onderzoeken overlappen gedeeltelijk met de bevindingen van dit proefschrift. Zo heeft eerder onderzoek ook een verband gevonden tussen \textit{Bacteroides} en beter inhibitievermogen. Er zijn echter ook verbanden gevonden die niet in voorgaand onderzoek naar voren kwamen. Verschillen tussen de resultaten van ons onderzoek en eerder onderzoek komt mede doordat er sprake is van verschillende leeftijden en gedragsmaten die zijn onderzocht. Meer (replicatie) onderzoek is nodig om de gevonden relaties te bevestigen en de bewijskracht te versterken.  

In \textbf{hoofdstuk 5} zijn de voorspellers van de kwaliteit van voeding van adolescenten onderzocht. Kwaliteit van zorg van de moeder is onderzocht op de kinderleeftijden van vijf weken, 12 maanden, twee-en-een-half jaar, 10 jaar en 14 jaar. Dit is onderzocht aan de hand van video's van interacties tussen moeder en kind die door onafhankelijke beoordelaars scores hebben gekregen. Voedingsinname en emotie-eten van adolescenten is middels zelfrapportage verzameld. Verder is het inhibitievermogen van de adolescent gemeten met behulp van drie gedragstaken en een vragenlijst die is ingevuld door de moeder. Er is geen bewijs gevonden voor maternale zorgkwaliteit en inhibitievermogen of dieetkwaliteit. Wel is er bewijs gevonden voor een verband tussen betere inhibitievermogen en betere dieetkwaliteit bij adolescenten. Toekomstig onderzoek zou (zelf)rapportage en objectieve observaties moeten combineren om een beter beeld te krijgen over kwaliteit van de zorg van moeders (en partners) en hoe dat relateert aan het gedrag van adolescenten (voedingsinname en inhibitievermogen). Langlopende en experimentele onderzoeken zijn nodig om de richting van de verbanden te achterhalen.  

Samenvattend, specifieke suikers in moedermelk en bepaalde darmbacteriën voorspellen betere executieve functies bij peuters. Tevens is er een verband gevonden tussen het inhibitievermogen en de dieetkwaliteit van een adolescent. Ondanks dat deze bevindingen geen oorzakelijke verbanden kunnen aantonen, wijzen de resultaten erop dat voeding in het vroege leven en de darmbacteriën mogelijk een rol spelen in de executieve functies en het inhibitievermogen van peuters. Daarnaast suggereren de resultaten dat er samenspel is tussen inhibitievermogen en voedingsinname tijdens adolescentie. 

\vspace{0.5cm}
\noindent Belangrijkste bevindingen van dit proefschrift:
\begin{itemize}
\item Langere borstvoedingsduur voorspelt niet het inhibitievermogen van een peuter, maar wel betere dieetkwaliteit van de peuter op de leeftijd van drie jaar.
\item Hogere concentraties van gefucosyleerde HMOs in moedermelk voorspellen betere executieve functies op peuterleeftijd.
\item We vonden geen bewijs voor een verband tussen gesialyleerde HMOs in moedermelk en executieve functies op peuterleeftijd.
\item Hogere niveaus van de darmbacterieën \textit{Streptococcus}, [\textit{Ruminococcus}] \textit{Torques} groep, \textit{Clostridium sensu stricto} 1, \textit{Intestinibacter}, en \textit{Halomonas} zijn gerelateerd aan verminderde executieve functies in peuters.
\item Hogere niveaus van de darmbacterieën \textit{Bacteroides}, \textit{Parabacteroides}, \textit{Ruminococcus} 2 en \textit{Blautia} en hogere diversiteit aan verschillende soorten darmbacteriën zijn gerelateerd aan betere executieve functies in peuters.
\item Hogere niveaus van de darmbacterieën \textit{Bacteroides}, \textit{Ruminococcaceae} UCG-013 en \textit{Veillonella} voorspellen beter inhibitievermogen in peuters. 
\item Hogere niveaus van de darmbacterieën \textit{Subdoligranulum}, \textit{Lachnospiraceae} NK4A136, \textit{Anaerostipes}, \textit{Sutterella} en \textit{Coprococcus} 3 voorspellen verminderd inhibitievermogen in peuters.
\item Beter inhibitievermogen van adolescenten is gerelateerd aan betere dieetkwaliteit.
\item Er is geen bewijs gevonden dat de zorgkwaliteit van de moeder tijdens de kindertijd de inhibitie en dieetkwaliteit van adolescenten voorspelt.
\end{itemize}
