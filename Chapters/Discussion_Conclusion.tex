\fancyhf{}
\fancyhead[C]{Chapter 6. General Discussion}% <- added
\fancyfoot[R]{\thepage\ifodd\value{page}\else\hfill\fi}
%\fancyhead[L]{\ifodd\value{page}\relax\else\hfill\fi Ch \thechapter}
%\renewcommand\headrulewidth{0pt}% default ist .4pt
\renewcommand{\plainheadrulewidth}{.4pt}% default is 0pt
\nohyphens{

\newpage
\noindent Executive functions (EF) and inhibitory control (IC) are essential skills that are important for executing goal-directed behaviours \citep{diamond_executive_2013}. Early life nutrition is suggested to have a pivotal role in explaining inter-individual differences in these behaviours \citep{costello_nutrients_2020}. The general goal of this thesis was to uncover unknown links between nutrition and behaviour in early life. We investigated this by means of three aims. The \textbf{first aim} was to investigate the relations between early life nutrition-related predictors and future cognitive and behavioural outcomes. The \textbf{second aim} was to investigate potential mechanisms underlying the relations between early life nutrition and cognitive and behavioural outcomes. Because of the important role of the microbiota-gut-brain axis on behaviour \citep{chakrabarti_microbiota-gut-brain_2022, cryan_microbiota-gut-brain_2019, morais_gut_2020}, we focused on the gut bacteria in early as well as later life (two weeks to three years of age). Lastly, as nutrition is important for the development of many physiological, and likely psychological systems \citep{norris_nutrition_2022}, it is important to study predictors of healthy nutritional behaviours, especially in phases of life where risk for unhealthy nutritional behaviours is heightened, such as adolescence. The \textbf{third aim} was therefore to investigate the role of maternal caregiving behaviour on adolescent nutritional behaviours.

In light of \textbf{aim 1}, we investigated whether the duration of breastfeeding predicted better toddler EF and IC, and if this was mediated by toddler diet quality (\textbf{Chapter 2}), and whether the human milk oligosaccharide (HMO) composition in breast milk at two, six, and 12 weeks predicted better toddler EF and IC (\textbf{Chapter 3}). We found that duration of breastfeeding, and toddler diet quality did not predict toddler IC and EF (\textbf{Chapter 2}). However, we did find a relation between longer breastfeeding duration and better toddler diet quality (\textbf{Chapter 2}). Furthermore, in \textbf{Chapter 3}, we found that higher concentrations of 2'FL and grouped fucosylated HMOs in mother milk predicted better toddler EF in exclusively breastfed toddlers. These relations were not found for 3'SL, 6'SL, and grouped sialylated HMOs.

For \textbf{aim 2}, gut microbiota composition at ages two, six, and 12 weeks, and one and three years, were assessed in relation to EF, IC, and problem behaviour in toddlerhood (\textbf{Chapter 4}). One of the most important findings was the association between higher relative abundances of \textit{Streptococcus} throughout the first three years of life and worse EF at age three years.

Lastly, for \textbf{aim 3}, we investigated the relation between maternal caregiving quality, measured from the early postnatal period until adolescence, and adolescent diet quality and emotional eating behaviour, and whether adolescent IC mediated these potential relations (\textbf{Chapter 5}). We found that higher diet quality was associated with better IC in adolescence. However, adolescent emotional eating behaviour was not related to adolescent IC. Finally, we did not find support for a relation between maternal caregiving quality (at all ages) and adolescent inhibitory control or diet quality and emotional eating behaviour.

Figure \ref{Figure 6.1} shows a summary of the results found. The following paragraphs discuss how these findings contribute to research on early life and development, as well as their limitations and directions for future research.

\begin{landscape}
\begin{figure}[h]
\centering
\includegraphics[scale=0.6]{Figures/Discussion_F1.pdf}
\captionsetup{justification=justified}
\caption{Overview of the research topics per chapter. Blue boxes indicate the expected predictors. The orange and green boxes indicate the outcome measures. Light orange dotted lines indicate the cohorts used. The black arrows indicate the expected direction of the association, as causality could not be determined in the present data. The bold green arrows indicate the associations found.}
\label{Figure 6.1} %necessary for refering to it in the text, but if I mentioned it here already, then it doesn't need to be changed.
\end{figure}
\end{landscape}

\section{The impact of early life nutrition on future behaviour}
\tab Early life has been shown to be an extremely important phase predicting future development \citep{robertson_human_2019}. In line with this, many studies investigating early life factors, such as early life stress and early life nutrition, have found impactful health consequences on children's future physiological and psychological wellbeing \citep{connor2019early, smith_early_2020}. In the current studies, we also found that early life nutrition predicted future child health outcomes. First, we found that longer exclusive breastfeeding duration predicted higher diet quality in toddlers (\textbf{Chapter 2}). This is in line with the results of many studies that found longer (exclusive and total) breastfeeding duration to be related to higher intake of vegetables and fruits in toddlers, contributing to a higher toddler diet quality (see reviews by \citet{ventura_does_2017} and \citet{ventura_maternal_2021}). A potential mechanism underlying these findings could be that the mother shapes healthier food preferences in the child through breastfeeding \citep{ventura_does_2017,ventura_maternal_2021}. Mother milk is the first exposure to flavours for most infants. The flavour of mother milk changes depending on the maternal diet. Specifically, compounds of garlic, carrot, anise, mint, eucalyptus, alcohol, and molecular structures of fruit and vegetables are transferred to human milk \citep{spahn_influence_2019}. An infant responds to this flavour change by increasing or decreasing sucking time and number of sucks \citep{mennella_transfer_1991}. Long-term studies show that during solid food consumption, young children show greater preferences (defined as greater duration of mouthing behaviour) for the flavours they have been exposed to through breast milk, with lasting effects until at least ten years of age (e.g., maternal diet with higher vegetable intake is associated with greater child preferences for, and intake of vegetables) \citep{mennella_prenatal_2001,sausenthaler_effect_2010,ventura_maternal_2021}. This explanation falls under the Lactocrine Programming hypothesis, which states that milk constituents can have long term effects on an infant's development (\citet{bartol_epigenetic_2008,hinde_daughter_2013}, for a review, see \citet{weerth_human_2022}). In this case, the effect would be a programming of the child's food preferences. Note that this proposed explanation is plausible under the assumption that mothers who breastfeed longer have healthier diets than mothers who breastfeed for shorter times \citep{amir_maternal_2012,beckerman_maternal_2020}. An alternative and most probable complementary explanation for our findings is that mothers who breastfeed longer also provide and feed their children healthier foods in toddlerhood.

A clear result in this thesis that is supported by the Lactocrine Programming hypothesis is the evidence we found for higher levels of fucosylated HMOs in mother milk predicting better toddler EF (\textbf{Chapter 3}). This is in line with rodent studies that confirmed a causal relation between early life HMOs and better cognition \citep{docq_protective_2020}. The review by \citet{docq_protective_2020} described rodent studies that administered HMOs in early life and assessed their behaviour in adulthood. In addition, several human studies also found relations between HMOs and better cognitive outcomes \citep{Berger_human_2020,Cho_2021,Jorgensen_2021}. When comparing the results of \textbf{Chapters 2} and \textbf{3}, they suggest that breastfeeding duration independently does not predict EF or IC, but that the constituents of mother's milk, i.e., HMOs, may be important for predicting cognitive outcomes. As there are still unclear relations between breastfeeding duration and child and toddler EF \citep{belfort_infant_2016, julvez_attention_2007,lopez_breastfeeding_2021}, both breastfeeding duration and milk-borne bioactive factors should be considered in future research to obtain a clear view on how Lactocrine Programming predicts toddler behaviour. Note, however, that causality cannot be established with observational longitudinal studies. Animal studies that include early life manipulations in experimental trials can unveil causal relations between early life nutrition and cognition, though results are still difficult to translate to humans \citep{Mudd_2017}. Hence, a combination of these methods (i.e., longitudinal observational human studies, experimental animal studies, randomised controlled trials), as well as replication studies to confirm previous results (to allow for performing meta-analyses), are necessary to determine whether and how early life nutrition causally affects future cognitive and behavioural outcomes in human children.

\section{The microbiota-gut-brain axis: mechanism underlying the relations between nutrition and behaviour?}
\tab We found that early life fucosylated HMOs predicted better toddler EF (\textbf{Chapter 3}). Interestingly, the main functions of HMOs are to serve as nutrition for the gut microbiota \citep{bode_functional_2015,Totten_2012,Underwood_2014}. Indeed, in the same BINGO cohort, HMO levels were positively related to higher levels of specific gut microbiota (e.g., high levels of 2'FL were related to higher levels of \textit{Bifidobacteria}) \citep{borewicz_association_2020}. In addition, animal studies as well as human studies have observed an interplay between the gut microbiota and brain functioning \citep{bundgaard-nielsen_gut_2020,cheung_systematic_2019,cryan_microbiota-gut-brain_2019,jiang_gut_2018,sukmajaya_systematic_2021}. Because of the potential regulatory function of the gut microbiota on the brain (e.g., through production of short chain fatty acids that affect the central nervous system \citep{dong_role_2022}), the microbiota-gut-brain axis is a strong candidate for explaining the relations found between early life nutrition and cognition. In line with previous findings \citep{carlson_infant_2017,guzzardi_maternal_2022,tamana_bacteroides-dominant_2021}, we found that high levels of \textit{Veillonella} and \textit{Bacteroides} in early life were related to better toddler IC (\textbf{Chapter 4}). In addition, findings came forth that were not seen in previous literature, e.g., we found that higher levels of gut \textit{Streptococcus} in early life as well as throughout the first three years of life, predicted worse toddler EF (\textbf{Chapter 4}). Although we speculated about potential pathways through which these gut microbiota may regulate behaviour, we were unable to confirm these pathways in the current thesis. To discover the causal role of gut microbiota in the relation between behaviour and cognition, future research should aim at identifying the function of bacteria, involving a multi-omics approach \citep{ou_microbiota-gut-brain_2023}, as explained below.

Metagenomics, including marker gene analyses (16S rRNA in bacteria) and shotgun sequencing, have allowed us to reveal the composition of bacteria in a stool sample \citep{almeida_new_2019}. Many studies identify the relative abundance of bacteria and infer on the relations found with the health outcome, leading to insightful discoveries, such as that bacterial composition between depressed, and anxious individuals differ significantly from healthy controls \citep{pinart_gut_2021}. Note, however, that interpretation of relative abundance data could lead to false discovery rates \citep{gloor_microbiome_2017}, and correlation biases \citep{tsilimigras_compositional_2016}. This is due to the fact that an increase in abundance of one taxa is equal to a decrease in all other taxa, meaning that relative abundance of one taxa is dependent on the abundance of the remaining taxa \citep{barlow_quantitative_2020}. Identifying absolute bacterial abundances (measured by e.g., quantitative PCR or flow cytometry) may aid in the interpretation of biological mechanisms regarding the microbiota-gut-brain axis. Importantly, as gut microbiota composition is subject to day-to-day, and within-person variation \citep{vandeputte_temporal_2021}, it is advisable to include repeated measures of stool sampling. To increase the reliability of earlier findings, replication studies are necessary \citep{peels_replication_2021}. When consistency in results is found over multiple studies, the next step is to explore the mechanism behind the relations.

Nonetheless, assessing relative or absolute abundance of bacteria does not reveal their transcriptional activities \citep{abu-ali_metatranscriptome_2018,granata_duodenal_2020,jia_rare_2019}. Metatranscriptomics, which is the identification of microbial mRNA, allows for identification of bacterial metabolic activities. Interestingly, this technique has helped uncover different functions of the same bacterial species. For example, in patients with Crohn's disease, \textit{Ruminococcus gnavus} is able to produce inflammatory glucorhamnan that induces production of inflammatory cytokines in the gut \citep{henke_ruminococcus_2019}. This same bacterium was also found to modulate mucin production which fortifies gut barrier functions \citep{graziani_ruminococcus_2016}, subsequently preventing gut inflammation \citep{luissint_inflammation_2016}. This phenomenon is likely explained by interactions between different bacterial taxa, through communication, cooperation, and competition \citep{coyte_ecology_2015,coyte_understanding_2019,zhang_quorum_2022}. This means that gut bacteria exert different effects depending on which other bacteria are present in the gut. Different functions of the same bacterial species could also be explained by the fact that bacteria exert different activities based on their host's conditions, also known as phenotypic switching \citep{sousa2012phenotypic,tadrowski_phenotypic_2018}. Interestingly, some mRNAs have weak ribosome binding sites and are therefore poorly translated, while those with strong binding sites are easier to translate \citep{liang_mrna_2000}. This means that not all microbial mRNA detected by metatranscriptomics may be involved in the expected metabolic processes, indicating that the interpretation of the results of these techniques must be done with caution.
 
Identifying microbial proteins (metaproteomics) and metabolites (metabolomics) might give a clearer picture of the role of gut bacteria in behaviour. Metaproteomics identifies levels of expressed proteins. The identification of these proteins is reliant on pipelines that process these data and match the peptides with online metagenomic databases to discover the most probable bacteria that might have expressed them \citep{mesuere_unipept_2016}. The reliance on these metagenomic databases is also a flaw, as they are dependent on the previously detected proteins. Hence, newly found peptides may not be in the database yet. Metabolomics is the study of metabolites in a biological sample, and allows for identification of key metabolites of specific pathways linked to a disease \citep{vignoli_high-throughput_2019}. However, as some metabolites are produced by different bacterial strains \citep{venegas_corrigendum_2019}, it is difficult to disentangle which bacteria are related to the identified metabolites. Note that including metaproteomics, and metabolomics result in more variables and therefore more potential interactions to be investigated. Hence, it is extremely important that this type of research is properly powered.

All of the above supports the notion that identification of relative abundance is just a first step to understanding the role of gut bacteria in behaviour. Although all omics-techniques may have flaws, these techniques can get us closer to identifying causality in the microbiota-gut brain axis. A multi-omics approach could therefore lead to a better understanding on how gut microbiota may play a causal role in the relation between nutrition and behaviour \citep{daliri_challenges_2021}.

Before validating the potential molecular mechanisms of bacteria, it is important to first isolate, culture, and characterise them. Although approximately only 20\% of the human gut bacteria have been cultured so far \citep{eckburg_diversity_2005}, it might be possible to culture hundreds of gut bacterial strains in a short timespan in the near future, due to rapid developments in high-throughput cultivation approaches \citep{clavel_next_2022}. Furthermore, besides bacteria, the gut is also colonised by fungi, archaea, and viruses. Fungi and archaea have different functions and have been associated with host phenotype, such as gut-related diseases and gut motility, respectively \citep{borrel_host-associated_2020,richard_gut_2019}. These microorganisms also interact with bacteria and their derived products \citep{borrel_host-associated_2020,richard_gut_2019}. Viruses present in the gut interact and coexist with gut bacteria through lysogeny (i.e., integration of the virus' nucleic acid into the bacterial genome or formation of a viral replicon in the bacterial cytoplasm). This way, viruses directly impact gut microbiota composition and the immune system, possibly modulating drivers of health and disease \citep{kirsch_bacteriophage-bacteria_2021}. Identifying the function of individual micro-organisms may be challenging. However, considering the complex interactions between the host, bacteria, fungi, and archaea, it is extremely valuable to obtain a detailed view of these dynamic interactions before validating these pathways.

Lastly, in vivo and in vitro validation studies can help confirm the expected pathways that are likely involved in the microbiota-gut-brain axis \citep{morais_gut_2020}. In vitro studies are studies outside of a (animal) body. For example, organoids, three-dimensional tissues cultures grown from stem cells, have been used more and more to model gut and microbial interactions \citep{moysidou_advances_2021}. In these models, microbes or their derived metabolites are injected into gut organoids to determine interactions between the gut and the microbiota \citep{moysidou_advances_2021}. In vivo studies model the potential causes of health outcomes, such as depression and anxiety, in animal models. These studies grant evidence for causal relations \citep{nagpal_microbiota-brain_2021,nestler_animal_2010}. For example, after transplanting microbiota of mice with high-anxiety into mice with low-anxiety, the behaviour of these recipient mice changed according to the donor's behavioural profile \citep{bercik_intestinal_2011}. Furthermore, transplantation of gut microbiota from humans with autism spectrum disorder into germ-free mice resulted in the induction of autistic behaviours in the mice \citep{sharon_human_2019}. However, interpretation of these human microbiota-associated rodent studies must be done with caution. Overall, the majority of these studies do not attempt to gain insight in the mechanisms (e.g., which genes are up-, or down-regulated after the stool transplant), and they use a small number of human donors \citep{walter_establishing_2020}. In addition, studies with null-results or negative outcomes are rarely published \citep{nissen_publication_2016}, and causal claims are overstated, also due to the fact that rodents are proxies for human diseases that do not occur naturally in rodents (e.g., what exactly are `autistic behaviours' or other behaviours in rodents?) \citep{freudenberg_challenges_2017,kazdoba_translational_2016}. Improving the experiments, and changing mindset and policies would aid in discovering true causality between gut microbiota composition and behaviour \citep{walter_establishing_2020}.
All in all, the steps for turning correlational relations into causal relations are many and complex. Nonetheless, it is clear that gut microbiota research would benefit from applying repeated measures, identifying absolute gut bacterial abundances, applying a multi-omics approach, and validating the pathways in in vitro and in vivo studies, to confirm causal relations between the gut microbiota and the brain. This approach, in combination with more specific hypothesis formation, larger sample sizes, and rigorous data collection, storage, and processing, will create evidence that can become substantial enough to result in future clinical implications.

\section{Maternal educational level: Key player contributing to child health}
\tab Maternal educational level was found to be important when investigating early life nutrition and later child behavioural outcomes (\textbf{Chapter 2} and \textbf{Chapter 5}). Higher maternal educational level correlated with later breastfeeding cessation age (\textbf{Chapter 2}), better adolescent inhibitory control, higher adolescent diet quality, and more adolescent emotional eating behaviour (\textbf{Chapter 5}). A vast majority of previous literature, related to this thesis' topic, included educational level in their analyses, and found it to impact their results \citep{belfort_infant_2016,julvez_attention_2007,kim_perceived_2008,kremers_parenting_2003,lopez_breastfeeding_2021,lytle_predicting_2003,snoek_parental_2007,strien_parenting_2019,vereecken_associations_2009}. Despite educational level being determined differently in each country and culture, study results regarding relations between educational level and child health are consistent. Higher parental educational level is consistently positively related to child physical and mental health \citep{fakhrunnisak_positive_2022,quesnel-vallee_socioeconomic_2012,wu_effects_2021}. There are some potential suggested mechanisms behind this relation. Parental education was shown to relate to improved parenting abilities and marital quality \citep{oreopoulos_priceless_2011}, as well as improved maternal ability to manage finances, choose the qualitatively best child educational programs, and control family health \citep{samarakoon_does_2015}. Furthermore, higher maternal educational level is usually accompanied by a higher income, allowing for provision of expensive healthier nutrition \citep{rippin_inequalities_2020}. One study also found that parental educational level affects child health through their own parental health and family living conditions \citep{wu_effects_2021}. As such, maternal educational level appears to impact child health in different ways, making it a key variable to include in future research.

In the current thesis, we observed potential effects of maternal educational level within a generally highly educated sample. Since lower educated and higher educated families differ in their dietary intake (i.e., lower educated families have less healthy diets) \citep{fard_interplay_2021,guerra-carrillo_does_2017}, it is imperative to include and retain lower educated families in future studies. As achieving this goal is unfortunately a common challenge for scientific studies, previous research has reported on the barriers and strategies that improve the inclusion and retention rates of lower educated families \citep{barnett_recruiting_2012,brannon_strategies_2013,nicholson_recruitment_2011,teague_retention_2018}. Generally, strategies to reduce participant burden (e.g., travel time, and flexible data collection methods) are most effective in retaining large sample sizes \citep{teague_retention_2018}. To reduce the burden in lower educated families, a higher budget may be necessary (e.g., in the form of a higher incentive, or time and labor of the staff to allow for stepped-interventions and rapport-building between staff and participants) \citep{barnett_recruiting_2012,brannon_strategies_2013,nicholson_recruitment_2011}. Summarizing, future research on nutrition and behaviour should include lower educated families. By applying the abovementioned strategies, future studies can discover and obtain consistent results from a more diverse sample (e.g., inclusion of a wider variety of educational levels), allowing for generalization of the results.

\section{Assessment of child behaviour: challenges and future directions}
\tab The assessment of behaviour is reliant on behavioural tests, (semi-)naturalistic observations of behaviour, and questionnaires, all of which have been applied in all our studies. At different ages, different types of measures may be more suitable than others for characterizing behaviour. Indeed, we found changing relations depending on the measure used, and age assessed (in \textbf{Chapters 3}, and \textbf{4}, we found results for toddler behaviour assessed with parental questionnaires by the primary caregiver, and in \textbf{Chapter 5}, we found results for adolescent behaviour assessed with behavioural tasks). This could be due to the fact that toddlers spend a large part of their time together with parents. Furthermore, toddler behaviour during a behavioural task could be variable as it may be dependent on their hunger levels, sleep quality, and the examiner performing the tasks (e.g., sex, examiner-toddler interaction) \citep{srinath_clinical_2019}. Adolescent behaviour is less affected by these factors \citep{srinath_clinical_2019}. However, as children become adolescents, they become more independent and desire more privacy \citep{sanders_adolescent_2013}, resulting in parents having a less clear view on their adolescent's behaviour. Adolescent self-report on behaviour had not been assessed in our studies due to time and burden constraints. However, it could be very valuable to include adolescent self-reports next to behavioural tasks, as, commonly, literature finds equivocal correlations between behavioural tasks and questionnaires \citep{toplak_executive_2008}. Both methods are valuable but also have their biases, with questionnaires being prone to socially desirable answers \citep{devriendt2009validity,stanton_prevalence_1996}, and behavioural tasks being a momentary assessment dependent on other factors, such as attention span \citep{markant_leveling_2014}. Hence, future research on toddler and adolescent behaviour ideally should include both behavioural tasks as well as (self-report) questionnaires. In case of time and/or budget restrictions, it may be advisable to assess toddler behaviour with parental questionnaires, and adolescent behaviour with behavioural tasks.

\section{Assessment of child dietary intake: challenges and future directions}
\tab Diet quality was determined by assessing the dietary intake with parental reports (\textbf{Chapter 2}, \textbf{3} and \textbf{4}) and self-reports (\textbf{Chapter 5}). The most commonly used methods for dietary assessment are 24-hour recalls, food frequency questionnaires, and food records \citep{amoutzopoulos_traditional_2018}. Although keeping a food record is currently the most valid method for assessing total dietary intake (i.e., `gold-standard'), the fact that the most commonly used dietary assessment methods are reliant on self-reports and parent reports, means that they are prone to memory bias as well as social desirability bias \citep{hebert_social_1995,subar_addressing_2015}. This makes the assessment of dietary intake in a general population one of the largest challenges in nutritional sciences. Additionally, assessing dietary intake in specific age-groups within a general population has its own specific challenges. Toddlers cannot report their own dietary intake, hence, we are dependent on parental reports. It is a logical choice, since toddlers spend most of their time with their parents, and parents provide the toddler with nutrition, allowing them to know exactly what their child consumes. However, this also makes parent report highly subject to socially desirable answers, as it is the parent who is responsible for the toddler's dietary intake. In addition, there are moments when the parents do not know the dietary intake of their child, such as when it goes to kindergarten or is cared for by babysitters (which is often the case in the Netherlands \citep{Nederlands_Jeugdinstituut_2022}). This makes parental report still challenging. Regarding adolescent dietary intake, as children become more independent \citep{sanders_adolescent_2013}, the parent's view on the dietary intake of their adolescent child becomes less clear. This means that self-reports are more representative and reliable for assessing adolescent dietary intake. However, the multiple 24-hour recalls needed to obtain a reliable picture of diet quality cause a relatively high participant burden. As recruitment and retention of adolescents in studies is already challenging \citep{jong_recruitment_2023}, a food frequency questionnaire, which is less burdensome than three unannounced 24-hour recalls, is more appropriate for assessing habitual dietary intake in adolescents. Note that self-reported nutritional questionnaires are also prone to socially desirable answers, recall bias, and misreporting in adolescents \citep{jones_misreporting_2021}. Fortunately, developments regarding the assessment of nutritional intake have been rising. Technology-based assessments, including image- and sensor-based technologies, are being developed and show promising results \citep{ho_validity_2020,kouvari_validity_2020,zhao_emerging_2020}. Specifically, image-based technologies might be interesting for assessing dietary intake in the general population, as participants can simply take a picture of their food before and after the meal with a smartphone. This method reduces subject bias, burden, and provides a more accurate view on portion size and type of food, compared to 24-hour recalls, food frequency questionnaires, and food records. The largest challenge with this technology is for algorithms to accurately identify the nutritional value of the food on the image \citep{dalakleidi_applying_2022}. More references, and thus, more time are necessary to improve this algorithm. All in all, dietary assessment methods are advancing, and technology-based assessments could aid in reducing the burden and improving the accuracy of reporting dietary intake. As the utilisation and development of technological image-based dietary assessment started roughly 10 years ago \citep{forster_online_2014,kristal_evaluation_2014}, it is likely that these technologies will be applied either as a stand-alone method, or in combination with traditional dietary assessment methods, in the foreseeable future. For the time being, the best way to assess dietary intake in toddlers is through parental reports. Specifically, multiple unannounced 24-hour dietary recall reports, although tedious, provide the most reliable data. Furthermore, adolescent dietary intake is most reliably assessed with a self-reported food frequency questionnaire.

\section{Conclusion and future perspectives}
\tab As reflected in this thesis, early life nutrition is an important life phase for future health and behaviour. We found evidence that supports the Lactocrine Programming hypothesis: fucosylated HMOs are related to better EF in our study. Furthermore, the microbiota-gut-brain axis may be an important mediator between nutrition and behaviour as we found certain bacteria in early life and throughout the first three years of life to be related to better EF and IC in toddlerhood. To identify the likely causal role of gut microbiota on behaviour, recommendations for future research are: performing repeated measures, identifying absolute gut bacterial abundances, applying a multi-omics approach, validating the pathways in in vitro and in vivo studies, and replicating studies. Additionally, including lower educational levels is imperative as the role of maternal educational level was already shown to be important in our generally highly educated sample. A higher budget is necessary to aid the inclusion and retention of this target group. Lastly, the methods for assessing nutritional intake have their flaws. However, it is likely that technological image-based assessment methods will be applied more commonly in the foreseeable future. This will allow for accurate assessment of dietary intake in the general population, resulting in more accurate inferences on the relation between dietary intake and health outcomes.

The development of EF skills is important for many future outcomes, such as academic performance, socioemotional competence, and general health \citep{ayduk_regulating_2000,duckworth_is_2013,eisenberg_effortful_2007,munakata_executive_2021,pascual_relationship_2019,robson_self-regulation_2020,shoda_predicting_1990}. This thesis contributes to understanding the role of early life breastfeeding duration and HMO exposure on toddler EF, and the possible regulating role of gut microbiota on toddler EF. Replicating this thesis' results will help confirm the roles of early life nutritional predictors for future child health and behaviour. These studies would add to the body of literature that shows the importance of early life predictors for future healthy diet and behaviour. In turn, this would provide input for policy makers and health care institutions aiming to improve an infant's early life exposures.}


\newpage
\begin{spacing}{1.0} % Set the line spacing to single spacing
\fontsize{8pt}{8pt}\selectfont
\bibliographystyle{apalike}
\renewcommand{\bibname}{References}
\bibliography{All_bibtex}
\end{spacing}