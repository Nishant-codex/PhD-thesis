\clearpage
\fancyhf{}
\fancyhead[c]{Chapter 8. Appendices}% <- added
\fancyfoot[R]{\thepage\ifodd\value{page}\else\hfill\fi}
%\fancyhead[L]{\ifodd\value{page}\relax\else\hfill\fi Ch \thechapter}
%\renewcommand\headrulewidth{0pt}% default ist .4pt
\renewcommand{\plainheadrulewidth}{.4pt}% default is 0pt
\section{Outreach activities}
 
\section{Brain Bee}

During my PhD, I also dedicated significant efforts towards scientific outreach and teaching. I was part of the Dutch Brain Olympiad (Brain Bee) from 2021 until 2024, which a national level Olympiad for high school students, encouraging them towards Neuroscience and Artificial Intelligence. During this time I helped out the Computational neuroscience and AI team in developing study materials for the participants and I also part of the organizing team for the Nijmegen chapter.  I have summarized my contributions to the read below:  

\subsection{Genetic Algorithms (Chapter 2.4)}

In this chapter, I introduced Genetic Algorithms (GAs) as a biologically inspired computational technique rooted in evolutionary principles such as natural and sexual selection. The chapter educates readers on how GAs model adaptation by iteratively improving a population of candidate solutions to solve complex optimization problems. Through clear explanations and detailed step-wise descriptions, I provided readers with:

\subsubsection{Biological Foundations:}

I explained the fundamental biological concepts behind evolution, highlighting natural selection and sexual selection as mechanisms that shape species adaptation. This laid the groundwork for understanding how these principles inform computational algorithms.

\subsubsection{Algorithmic Steps and Mechanics:}

I clearly outlined the stages of a GA   from initializing a population, through selection, crossover (recombination), and mutation operators, culminating in termination upon convergence. This was supported by illustrative examples such as evolving binary strings, which concretely demonstrate how offspring solutions combine and mutate parental traits.

\subsubsection{Fitness Function Concept:}

Emphasizing the critical role of the fitness function, I described how it quantifies the quality of candidate solutions relative to the problem environment, guiding selection and evolution toward better solutions.

\subsubsection{Applications in Neuroscience:}

To connect theory with practice, I discussed real-world uses of GAs in neuroscience, such as fitting complex conductance-based neuron models to electrophysiological data and characterizing ion channel kinetics, emphasizing GAs' strength in handling high-dimensional parameter spaces.

\subsubsection{Limitations:}

I also covered computational constraints of genetic algorithms, including their expense in time and resources, occasional lack of guaranteed optimality, and sensitivity to implementation choices like selection criteria and mutation probabilities, providing readers with a balanced perspective.

\subsubsection{Educational Features:}

The chapter includes a flowchart to visualize the GA process, key terms for review, and pointers to further learning materials including interactive simulations and videos, making it accessible for learners preparing for neuroscience competitions or seeking foundational knowledge.

\subsection{AI and Neuroscience Applications Outside the Brain: Ethics and Impact (Chapter 2.5)}

This chapter presented a broad, interdisciplinary exploration of how insights from neuroscience and AI intersect in societal contexts beyond brain research, highlighting ethical considerations and future impacts:

\subsubsection{Real-World AI Applications:}

I detailed diverse societal domains where neuroscience-inspired AI is applied, such as neuromarketing using brain activity to predict consumer behavior, consciousness research that probes the neural basis of subjective experience, and sleep studies enhanced by AI analysis of brain signals.

\subsubsection{Neurotechnology and Neuro-Robotics:}

The chapter introduces next-generation technologies including neuro-robotics, where biological neural models guide autonomous robot behavior, and neuromorphic computing, aiming to build human-brain-like energy-efficient spiking neuron-based processors. This cutting-edge content situates the reader at the frontier of AI-neuroscience innovation.

\subsubsection{Ethical Challenges:}

Emphasizing the ethical dimension, I discussed critical concerns posed by neurotechnologies (e.g., unknown long-term impacts, regulatory gaps), privacy of brain data, the limitations and risks of lie detection via fMRI, cognitive enhancement through drugs, and pressing dilemmas such as the moral algorithms in autonomous vehicles and judicial predictive software highlighting societal risks of bias and fairness.

\subsubsection{Societal Impact and Sustainability:}

I highlighted the transformative impact of AI-driven automation on employment, the cognitive effects of digital systems on users, and the urgent need to address the growing energy footprint of computational infrastructures globally. This holistic perspective prepares the reader to think critically about the promises and perils of emerging technologies.

\subsubsection{Didactic Structure and Resources:}

The chapter integrates learning objectives guiding reader focus, defines specialized terminology, and points to references and further reading to encourage deeper inquiry into ethical AI and neuroscience's societal roles.

\subsubsection{Overall Contribution Impact}

My work brings conceptual clarity, scientific rigor, and ethical awareness to the reader, equipping learners with a detailed understanding of genetic algorithms a central evolutionary computation tool and a thoughtful exploration of AI \& neuroscience beyond the laboratory. By blending foundational science with real-world examples and interdisciplinary viewpoints, I contributed substantially to making complex topics approachable and relevant for the next generation of computational neuroscientists and AI researchers. Moreover, this endeavor taught me how to simplify complex topics to accessible to a laymen audience which is not at all an easy task. It was a fruitful and compelling experience overall. 

 
\newpage
