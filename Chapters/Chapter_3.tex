\fancyhf{}
\fancyhead[C]{Chapter 3. Neuromodulatory Control of Cortical Function:
Cell-Type Specific Reshaping of Neuronal
Information Transfer}% <- added
\fancyfoot[R]{\thepage\ifodd\value{page}\else\hfill\fi}
%\fancyhead[L]{\ifodd\value{page}\relax\else\hfill\fi Ch \thechapter}
%\renewcommand\headrulewidth{0pt}% default ist .4pt
\renewcommand{\plainheadrulewidth}{.4pt}% default is 0pt


\newpage
\section{Abstract}
\tab 
Neuromodulatory systems modulate circuit flexibility by acting on neuronal properties, yet their cell-type-specific influence on functional identity and stimulus encoding remains unresolved. Here, we assess how dopaminergic (D1R, D2R) and cholinergic (M1R) receptor activation alters the physiology and information processing of excitatory and inhibitory neurons in the mouse somatosensory cortex. Using whole-cell recordings under a Frozen Noise protocol, we measured four core functional domains—action potential dynamics, passive membrane properties, adaptation currents, and spike-triggered input filters. Receptor activation reorganized correlations among these features, with excitatory neurons exhibiting reduced coupling and inhibitory neurons showing increased coherence. Unsupervised clustering revealed that neuromodulation reshaped the distribution of functional neuron types. Critically, receptor activation modulated the transfer of stimulus information in a receptor- and cell-type-specific manner, reducing fractional information transfer in excitatory neurons. Our findings demonstrate that neuromodulators restructure the computational landscape of cortical circuits, tuning both functional identity and encoding fidelity.

\newpage
\section {Introduction}


Neuromodulators such as dopamine has been known to play a role in memory, learning and other higher cognitive function in the brain and has also been implicated with diseases such as Parkinson's and schizophrenia (\cite{durstewitz2008dual,arnsten2012neurobiological,winterer2004genes}), similarly acetylcholine has been implicated with brain function such as learning, sleep and attention \cite{dalley2004cortical,hasselmo2006cholinergic,sarter2009phasic}. These neuromodulators are known to affect the excitability of neurons in various parts of the brain \cite{bargmann2012beyond,bargmann2013connectome,marder2012neuromodulation,taghert2012peptide}. Dopamine and acetylcholine alter the intrinsic and tuning properties of neurons by activating G-protein-coupled receptors (GPCRs) such as Dopamine D1 and D2 receptors and muscarinic acetylcholine M1 receptors \cite{dascal2001ion}, which differs from neuron to neuron due to the distribution of these receptor types \cite{nusser2009variability} in each neuron. Activation/inactivation of these receptors have long been known to modulate excitability and firing properties (for example, D1R activation has been found to increase firing rate of excitatory neurons in PFC \cite{seong2012d1}, similar effects have been recorded for D2R \cite{cousineau2020dopamine} and M1R). However, despite ion-channel and behavioral understanding of receptor-specific modulation, it is still unknown how functional properties, which ranges from passive to filtering properties of individual neurons are altered by the activation of specific receptors in a population.    

Since response to synaptic input depends upon the passive properties such as conductance of the neuron \cite{hausser2000hodgkin} which ultimately depends on activation level of active ion channels at any given instant, modulation of passive properties ultimately modulates the response to synaptic input i.e, the receptive field \cite{ferguson2020mechanisms,salinas2001book}. Although, these neuromodulators change the action potential dynamics of the cell through the activation/inactivation of ion channels, that results in changing the broad scope properties including information transferred to the postsynaptic neuron. Traditional approaches that study neuromodulation have largely focused on gross excitability measures such as firing rate or spike frequency adaption \cite{nadim2014neuromodulation,marder2012neuromodulation,shine2021computational}. Although these measures are informative, it under represents the high dimensional space in which neurons operate. Neurons integrate inputs over time, encode information via complex nonlinear transformations of the synaptic input, and exhibit diverse passive and active membrane properties. It is still not completely understood how high-dimensional functional properties such as action potential dynamics, adaptation, passive biophysical attributes and input feature selectivity is altered by specific neuromodualtory signal. More specifically, it is still unknown how these functional properties change as a result of Dopamine (D1R and D2R activation) and acetylcholine (M1R activation) and how this high-dimensional modulation varies on cellular and population level and how it ultimately affects the amount of information transferred by a neuron to its postsynaptic neuron. 

In this study we address the question: how high-dimensional neuronal functional landscape is altered by Dopamine and Acetylcholine and what is the implication of this modulation on encoding properties of neurons? In order to understand how activation of D1R, D2R and M1R alters the functional properties and ultimately information transfer in a population of neurons, we analyzed in-vitro single neuron electrophysiological recordings under a Frozen Noise protocol from layer 2/3 in the somatosensory cortex in mice in sequence, first under a control (artificial cerebral spinal fluid (aCSF)) and then with a receptor agonist trial. We extracted four sets of functional features namely 1.) action potential (AP), 2.) passive biophysical (PB), 3.) adaptation current (AC) and 4.) linear input filter via a spike triggered average (STA). 

To address the question we first compared the coordinated change in passive and input driven as a result of neuromodulation using a multi-set correlation and factor analysis (MCFA). We further employ (UMAP) combined with Louvain clustering to explore how neurons reorganize into distinct functional subgroups depending on the neuromodualtory context. Since our final aim was to understand the consequence of neuromodualtion on function, we measured the amount of transferred information about input in the spike train during control and agonist trials and compared it across E/I cell types to understand the heterogeneous effect of modulation. In summary, our results provide a broad understanding of neuromodulation on functional properties of neurons in somatosensory cortex. By bridging the gap between receptor-level neuromodulation and high-dimensional functional phenotyping, these results provide a framework for modeling and studying neural circuits and for further exploring how information processing is affected as a result of cell-type specific neuromodulation and modeling disorders.                    

    
To address the question we first studied the population level reorganization by neuromodulation by comparing how D1, D2 and M1 agonist alters classification using an unsupervised high-dimensional UMAP+Louvain clustering approach based on AP, PB, AC and STA compared to a control (aCSF). This allowed us to study if neuronal population is heterogeneously modulated or classify into subgroups or uniformly with specific receptor activation. We then compared the coordinated change in passive and input driven as a result of neuromodulation using a multi-set correlation and factor analysis (MCFA). To finally understand the functional affect of specific neuromodulator we compared the transferred information about the stimulus in the somatic input and the spike train in control and agonist trials for excitatory and inhibitory neurons. In summary, our findings provide a systems-level view of how neuromodulation reorganizes intrinsic and input-driven properties of neurons. By linking receptor-specific activation to high-dimensional changes in neuronal function and information encoding, this work offers a principled framework for modeling neuromodulatory effects in neural circuits and understanding how such modulation might contribute to dynamic brain states and disease phenotypes.


\section{Methods}

\label{methods}

\textbf{Ethics statement} The data used in this research was previously published and made freely available to the community \cite{da2018databank} and \cite{yan2022whole}. All the experimental work, as outlined in the cited articles, were carried out in compliance with the European directive 2010/63/EU, the national regulations of the Netherlands, and international standards for animal care and use.    

\begin{flushleft}
\textbf{Slice electrophysiology} Data acquisition procedures, the details of the in vitro slice preparation, intracellular access to anatomically targeted neurons, data digitization, and preprocessing have been described in detail elsewhere \cite{kole2020assessing,kole2019neocortical,da2018databank,kole2017proteomic,miceli2017reduced}. In short, Pvalbtm1(cre)Arbr (RRID:MGI:5315557) or Ssttm2.1(cre)Zjh/J (RRID:IMSR\textunderscore JAX: 013044) mice, including both females and males, were obtained from local breeding colonies and studied after the maturation of evoked neurotransmitter release in the primary somatosensory cortex (\cite{martens2015developmental}).\\
\vspace{0.25 cm}
Mice were anesthetized with Isoflurane (1.5 mL/mouse) before extracting tissue, and coronal slices of the primary somatosensory cortex (barrel subfield) were prepared.  The brain was removed, and 300 µm-thick coronal slices were made. Slices were then incubated in artificial cerebrospinal fluid (aCSF) (120 mM NaCl, 3.5 KCl, 10 glucose, etc.), aerated with 95\% $O_2$/5\% $CO_2$ at 37°C, and then at room temperature after 30 minutes.\\
\vspace{0.25 cm} 
Whole-cell electrophysiological recordings were performed with continuously oxygenated aCSF. The barrel cortex was localized, and cells in the supragranular layers were patched under 40x magnification using HEKA EPC 9 and EPC10 amplifiers with Patch Master software.  Patch-clamp electrodes were pulled from glass capillaries (1.00 mm external diameter, 0.50 mm internal diameter) and used with 5–10 MOhm resistance, filled with intracellular solution (130 mM K-Gluconate, 5 $KCl$, 1.5 $MgCl_2$, etc., pH adjusted to 7.22 with $KOH$). Data were band-pass filtered at 0.1–3000 Hz before storage for offline analysis.
\end{flushleft}



\begin{flushleft}
\textbf{Frozen Noise (FN) protocol} The Frozen Noise input protocol consisted of injecting a somatic current that is the result of an artificial neural network of 1000 neurons responding (firing Poisson spikes) to random stimuli i.e., the hidden state, the membrane potential response to the somatic input is recorded with a sampling rate of 20 kHz for a total length of 360 seconds and saved. Each raw data file consisted of a vehicle control trial (artificial Cerebrospinal fluid i.e. aCSf)  and a drug trial (a specific neuromodulatory receptor agonist or antagonist was added to the bath and the recording was repeated). Some files consisted of multiple control and drug trials. See \cite{zeldenrust2017estimating,da2018databank} for more details.    
\end{flushleft}

%%%  Please list here under separate headings 
%%%  all the experimental models/study participants 
%%%  (animals, human participants, plants, microbe 
%%%  strains, cell lines, primary cell cultures) 
%%%  used in the study. For each model, provide 
%%%  information related to their species/strain, 
%%%  genotype, age/developmental stage, sex (and 
%%%  gender, ancestry, race, and ethnicity if 
%%%  reported for human studies), maintenance, 
%%%  and care, including institutional permission 
%%%  and oversight information for the studies 
%%%  the experimental animal/human participant 
%%%  study. The influence (or association) of sex, 
%%%  gender, or both on the results of the study 
%%%  must be reported. In cases where it cannot, 
%%%  authors should discuss this as a limitation 
%%%  to their research’s generalizability.

%%%  Please omit this component if your study does 
%%%  not use experimental models typical in the 
%%%  life sciences (e.g., if your study is 
%%%  computational or physical science research). 

\subsection*{Method details}

\subsubsection*{Feature Extraction} 
\begin{flushleft}
We extracted waveform, action potential, passive biophysical and spike triggered average from single neuron recordings recorded under Frozen Noise input protocol, these recordings were performed first under a vehicle control (aCSF) and then repeated with a receptor agonist added to the bath. In total 288 neurons (Table. \ref{tab:neuron_table}) were analyzed, we discarded recordings with high levels of noise.   

\begin{table}[!htb]
    \centering
    \begin{tabular}{cc}
 \hline
 Condition&Trials\\
 \hline
         aCSF& 288\\
         D1& 54\\
         D2& 59\\
         M1& 41\\
\hline         
    \end{tabular}
    \caption{Number of neurons with agonist applied}
    \label{tab:neuron_table}
\end{table}

\end{flushleft}

\begin{flushleft}
\textbf{Spike waveforms} As explained in \cite{joshi2024understanding}, we identified peaks from the membrane potential traces and kept the hyperparameter and ISI threshold criteria the same as in \cite{joshi2024understanding}. The length of waveforms used in this study is 10ms (5ms before and after the peak).      
\end{flushleft}

\begin{flushleft}
\textbf{Action potential features} 
The action potential features were extracted to study the dynamics, threshold and waveforms related features throughout a trial via descriptive statistics.  The action potential features were extracted for aCSF and agonist trails as described in \cite{joshi2024understanding}. 
\end{flushleft}


% \newpage
\subsubsection*{Biophysical Feature extraction using GLIF model}
In order to extract passive biophysical attributes as well as adaptation current from aCSF and agonist trials, we fit a Generalized Leaky Integrate and Fire (GLIF) neuron model  \cite{pozzorini2015automated} as described in detail in \cite{joshi2024understanding}. We take the 100 second instance from trial as the training set extract passive features as well as adaptation current from the trial.  


\paragraph{Spike Triggered Average} The spike-triggered average (STA) is the average shape of the stimulus that precedes each spike. We extracted the STA using the following equation given by \cite{schwartz2006spike}: 
\begin{equation}
STA = \frac{1}{N}\Sigma_{n=1}^{N}  \overrightarrow{s}(t_{n}),    
\end{equation}
\begin{flushleft}
where $t_n$ is the $n^{th}$ spike time, s is the stimulus vector preceding the spike for a fixed time window of 100 ms, and N is the total number of spikes. Before clustering, we standardize (i.e. z score) and then normalize the STA vector with an $L_2$ norm. We didn't use any kind of whitening or regularization to calculate the STA.      
\end{flushleft}


% \subsubsection{\textbf{CLUSTERING METHOD}}
\subsection*{UMAP + Louvain clustering} 
Universal Manifold Approximator (UMAP) is a non-linear dimensionality reduction algorithm which is advantageous for preserving global structure of the data \cite{mcinnes2018umap} in lower dimensions, this makes it more suitable for visualization especially for high dimensional datasets ($p>>N$, where p is the dimensionality of the data and N is number of samples) compared to other methods such as PCA which fail to perform due to curse of dimensionality \cite{aggarwal2001surprising}.  As explained in \cite{lee2021non,joshi2024understanding} the high-dimensional graph obtained during the intermediate step in the UMAP algorithm can be exploited to perform clustering using Louvain community detection \cite{blondel2008fast}. We chose the hyperparameter based on cluster stability criteria and the corresponding number of clusters based on the same heuristic as explained in \cite{joshi2024understanding}.  

For measuring clustering similarity between aCSF and drug conditions, we calculate a cluster similarity matrix as explained in \cite{joshi2024understanding}. For quantifying the similarity between labels assigned to aCSF versus drug trials, we used cluster similarity measure such as adjusted random index and adjusted mutual information score using scikit-learn Python package \cite{scikit-learn}.     

\subsection{Information transfer protocol}
We used the information transfer protocol described in detail in \cite{zeldenrust2017estimating, zeldenrust2024tuning}, this method involves calculating entropy of the hidden state x which can be a stimuli that is observed by the network. Further steps involves calculation of mutual information between the hidden state and the input current $MI_I$ and again between the hidden state and the spike times $MI_{spike times}$ for a given neuron. The fraction of mutual information, referred to as fraction of information ($FI$) gives a measure of how much information is transferred by the spikes about the hidden state for a given neuron.
\begin{equation}
    FI = \frac{MI_{spike times}}{MI_I}
\end{equation}
Since the entropy of the hidden state will always be greater or equal to mutual information values. The value of FI will always be between 0 and 1. This calculation works under the assumption of ergodicity, therefore, an average over samples is the same as an average over time. This method also assumes that spike trains are Poissonian. The method still works, if there are minor deviations from Poisson distribution. 

\subsection*{Quantification and statistical analysis}
We used the non-parametric Wilcoxon test to test significance difference in information transfer (FI) between paired aCSF and drug trials . We also used one sided Student's T-test to test significance change between aCSF and agonist condition for passive biophysical and action potential attributes. Significance value was set to $p<0.05$ in both cases.  We performed Kruskal-Wallis H test and a post-hoc Mann-Whitney U test with Bonferroni correction for multiple comparisons for comparing cosine similarity between aCSF and agonist trials. All statistical tests were performed using scipy-stats package \cite{2020SciPy-NMeth}.  
 

\subsection*{Multi-set Correlation and Factor Analysis}

In order to assess the correlation between AP, PB, AC and STA attribute sets extracted from the electrophysiological recordings, and to further analyze how the correlation changes as a result of the application of agonists we used Multi-set correlation and factor analysis (MCFA) \cite{brown2023multiset} as explained previously in \cite{joshi2024understanding}. It is an unsupervised integration method designed to analyze multiple high dimensional data types from the same sample. This method is a combination of multi-set canonical correlation and factor analysis. The outcome is a joint model of the shared and private space between these high dimensional datasets. We used this method to extract the shared and private variance for each dataset and compare how these variances change as a result of the application of a certain agonist compared to a control condition. 

We center and scale all datasets and initialize the loading matrices similar to the \cite{brielin_brown_2023_8128339,joshi2024understanding}, the number of principal components for aCSF condition were chosen based on the Marchenko Pasteur Law and the size of private space $k_m$ for AP, AC and STA datasets were chosen to be 2 and 1 for the PB dataset. For agonist condition, instead of choosing the PCA components using the Marchenko Pasteur Law, we chose the number of principal components to be 2 and $k_m$ to 1 for all four datasets. This was done due to relatively small sample size for excitatory and inhibitory agonist sets. The fit is performed using an expectation minimization algorithm in order to obtain the shared and private loading matrices as well as shared and private spaces as described in \cite{joshi2024understanding}.     

\section{Results}

To understand how neuromodulation affects the high dimensional functional landscape dynamics of neuronal population we extracted four attributes from single neuron in-vitro somatic recordings \cite{yan2022whole}, performed using the frozen noise protocol (see \ref{methods}), namely action potential (AP), passive biophysical (PB), adaptation current (AC) and linear input filter approximated using a spike triggered average (STA) as described in (see \cite{joshi2024understanding} and \ref{methods}), we extracted these features from 296 distinct sets of neural recordings and divided them into excitatory and inhibitory subsets based on their waveform shapes and firing rates as described in \cite{joshi2024understanding}. We then separated the recording sets based on the receptor agonists applied and the corresponding aCSF trials. Here we present the analysis based on D1-agonist, D2-agonist and M1-agonist trials. 

\begin{table}[ht]
\centering
\begin{tabular}{ll}
\hline
\textbf{Functional Attributes} & \textbf{Description} \\ \hline
Action Potential (AP) & An ensemble of descriptive statistics of action potential   \\
&                    shapes and dynamics. \\
Passive Biophysical (PB) & Attributes related to the non-active properties of cells, \\
& such as membrane capacitance and resistance. \\
Adaptation Current (AC)& Refers to the ionic currents in neurons that change \\
& in response to prolonged stimuli. Extracted via fitting \\
& a GLIF model.  \\
Spike Triggered Average (STA) & The average of signal features occurring before neuron \\ 
& spikes, used to understand stimulus-response relations. \\
& It is an approximation of the linear input filter of a neuron. \\ \hline
\end{tabular}
\caption{Functional Attributes and their Descriptions}
\label{tab:functional_features}
\end{table}

\subsection{Neuromodulation changes structured correlation between functional attributes in a cell-type as well as receptor type specific manner}


We wanted to understand how neuromodulation (dys)regulate AP, PB, AC and STA attributes which represent distinct modalities of functional landscape. To study the conjoint effect of neuromodulation on active and input driven functional attributes and how the correlation between attribute sets change as a result of specific receptor activation, we applied multi-set correlation and factor analysis (MCFA) to our dataset, first on aCSF (control) trials and each agonist trial set respectively. We summarized the results of shared, private and residual variance using stacked histogram (Fig. \textbf{\ref{fig:mcfa}}). In this analysis, when individual feature sets exhibit high shared variance, this indicates that the features are functionally coordinated: changes in one set tend to co-vary with changes in others. In contrast, when feature sets exhibit high private variance, it suggests that the features are relatively independent, meaning changes in one set do not systematically correspond to changes in the others. Or a feature with high private variance varies independently from the other features. A high residual signifies either noise or a complex non-linear relationship not captured by a linear method such as MCFA. 

It can be seen from Fig. \ref{fig:mcfa} that the relationship between functional attributes is cell-type specific. Comparing the histograms for aCSF trials (Fig. \textbf{\ref{fig:mcfa}}\textbf{A} ) between excitatory and inhibitory population, we can see that the private variance for Linear Input Filter attribute (STA) is higher for inhibitory neurons (see Table.\ref{tab:specific_values_inh}) compared to excitatory neurons (see Table.\ref{tab:specific_values_exc}), this shows that STA in inhibitory population is not strongly correlated with other functional attributes, therefore STA in the inhibitory population is not coordinated with other functional properties. The passive biophysical and action potential attributes show a high shared variance for excitatory neurons compared to inhibitory neurons. This show a high level of coordination between AP and PB attributes between excitatory neurons compared to inhibitory neurons.         

It can be seen from fig. \ref{fig:mcfa} that private variance for excitatory population vanishes for AP, PB and AC sets except for STA, for which it increases drastically compared to aCSF trials (see Table.\ref{tab:specific_values_exc} and Figure \ref{fig:mcfa} \textbf{A-B}) upon D1 receptor activation. The shared variance reduces sharply compared to aCSF for AC and STA sets for the excitatory population. This suggests that D1R activation makes STA less functionally coordinated with other attributes. While the shared variance for AP an PB are slightly reduced suggesting that the coordination level between these two properties remain intact. We observe a dramatically different effect in inhibitory population compared to the excitatory counterpart. Private variance decreases sharply for AC and STA sets (see Figure. \ref{fig:mcfa}\textbf{A-B}and Table. \ref{tab:specific_values_inh}), while drastically increasing the shared variance for these attributes, suggesting that AC and STA become functionally coordinated with the latent space, therefore AP and PB properties. In summary, effect of D1R is cell-type specific and nuanced. Increasing coordination between properties in inhibitory neurons while decreasing coordination for excitatory linear-input filter.           
In a similar manner, in order to study the effect of D2R activation, we analyzed D2 trials using MCFA and present in comparison with aCSF trials (see Fig. \ref{fig:mcfa}\textbf{C} and Table. \ref{tab:specific_values_exc}. We observe that D2 modulates the coordination between functional attributes in more subtle manner for excitatory neurons, the shared variance for PB and STA decreases, while opposite effect is seen in case of AC and AP attribute sets. Private variance on the other hand decreases drastically for AP and AC attributes as also observed in case of D1 and increases for STA and PB. The stark increase in private variance in case of STA needs to be highlighted. This suggests that D2 makes AP and AC properties more coordinated, while decreasing coordination between STA and PB attributes. In case of inhibitory neurons, the modulation is rather straight forward. The private variance for all the four attributes decreases and the shared variance increases for all the attributes except for AP attributes. This suggests that D2R overall makes the functional properties more coordinated for inhibitory similar to D1.            
In case of M1, we observe an entirely different functional landscape. The shared variance increases for AC an AP properties similar to D2 for excitatory neurons and decreases for PB  and STA. This suggests that AC and AP attributes become coordinated and PB and STA properties become more independent as a result of M1R activation. On the other hand, private variance decreases for all the attributes except for PB. For inhibitory neurons, the shared variance increases for all sets except for AC, suggesting a stronger functional coordination. Surprisingly, the private variance increases sharply for AC suggesting a decoupling with the rest of the attributes set.    

For a broader understanding of neuromodualtion, specifically D1R, D2R and M1R activation, how it affects the coordination structure between functional attribute sets, we calculated the average shared over private variance for each agonist as well as aCSF conditions, we observe that the fraction of shared over private variance increases for D1R, D2R and M1R trials compared to aCSF trials for both excitatory and inhibitory populations (Fig. \textbf{\ref{fig:mcfa}}\textbf{E}). This suggest that on average functional attributes become more coordinated as a result of specific receptor activation. 

In summary, we observed that neuromodulation changes the coordination structure between functional properties in a cell type as well as agonist specific way. For excitatory neurons the individual variability diminishes compared to their inhibitory counterparts as seen in Fig. \ref{fig:mcfa}\textbf{E}. In case of inhibitory population, on average the neuromodulation increases coordination between functional attributes.  


\subsection{Specific receptor activation alters functional classification }

Neuromodualtors are known to affect the intrinsic properties as well as excitability of neurons. Such as, it well understood that D2-R activation modulates excitability in motor cortex \cite{cousineau2020dopamine}. Neuronal functional classification which is classically studied using action potential and passive biophysical features, has been shown to change as a function of input to the neuron \cite{joshi2024understanding, hernath2019alternative}. Also, we have shown in the previous section that neuromodulation (D1, D2 and M1) changes the correlation structure (measured using shared variance) between functional attributes as well as individual private variance for each functional attribute set in a cell-type specific manner. Therefore it is important to understand how functional attributes (intrinsic, excitability as well as dynamics) are altered by D1-R, D2-R and M1-R activation as well as how neuronal clustering based on functional attributes changes, if at all, as a result of neuromodulation.   

To this aim we first wanted to check if there is a recoding drift present in the data as result of experimental setup when performing multiple trial, for this we extracted the 4 functional attribute sets along with their waveforms from experiments with multiple aCSF trials and compared the clustering between aSCF trial 1 with trial 2. The histogram in Fig. \ref{fig:S3} show that there is a high level of correspondence between aCSF trial 1 and trial 2 for waveform and passive biophysical features. The low level of correspondence for the AP, AC and STA results from the differences in inputs that cells receive in the two trials. We also show the manifold overlap between trial 1 and trial 2 for all properties (see Fig. \ref{fig:S3}).   

We clustered D1-R, D2-R and M1-R agonist group trials as well as their corresponding aCSF trials separately using UMAP+Louvain clustering (see Methods as well as \cite{joshi2024understanding}) and measured the similarity in clustering using adjusted mutual information score (AMI) see \ref{methods}. We summarized the clustering for excitatory neurons in case of D1-R activation in histogram (Fig. \textbf{\ref{fig:clustering} A.1}), it can be seen that AMI scores are consistently low for AP, PB, AC and STA, suggesting that functional attribute based clustering is altered as a result of D1-R activation. We further explored how each attribute set is altered as a result of D1-R activation, for passive biophysical properties (Fig. \textbf{\ref{fig:clustering} A.2}) we found that conductance (gL) (One Sided t-test: p<0.001) and reset voltage (Vr) (One Sided t-test: p<0.05) are significantly reduced as a result of D1-R activation. We then explored how adaptation current and linear input filter (STA) are altered as a result of D1R modulation shown in Fig. \textbf{\ref{fig:clustering} A.3} and Fig. \textbf{\ref{fig:clustering} A.4}, the red curves represent D1 trials and black curves represent aCSF trials. To quantify the differences between D1 and aCSF trials, we calculated the rise time and peak values (see \ref{methods}) for both adaptation currents and STA. The joint plot Fig. \ref{fig:S6} shows the rise time and peak differences between D1 and aCSF trials for adaptation current and Fig. \ref{fig:S7} shows rise time and peak value for STA. The peak and rise time for adaptation current (AC paired t-test (decay time): p = 0.4341, AC paired t-test (peak): p = 0.2444) and STA (STA paired t-test (rise time): p = 0.758, STA paired t-test (peak): p = 0.0514) were found to not be significantly different between aCSF and D1 trials for excitatory neurons. This suggest that STA and adaptation currents are not altered significantly as a result of D1-R activation. We also performed a cosine similarity measurement within and across the aCSF and D1 trials for AC curves, we performed a Kruskal--Wallis H test on three groups: within-aCSF, within-agonist, and across-aCSF and agonist conditions. The test revealed a significant effect of group on similarity distributions ($H(2) = 15.3953$, $p = 4.53e-4$). Post hoc comparisons using Mann--Whitney U tests (Bonferroni-corrected for multiple comparisons) showed that:
\begin{itemize}
    \item Similarity scores were significantly higher within aCSF compared to across-pair comparisons ($U =120654.00 $, $p = 3.17e-4$),
    \item D1 also showed significantly lower similarity than across-pair comparisons ($U = 153267.00$, $p = 0.0217$),
    \item We didn't observe a significant difference between aCSF and D1 mean cosine similarity distributions ($U = 135299.00$, $p = 1.0$).
\end{itemize}

These findings indicate that neural representations are more consistent within conditions than across conditions, suggesting that D1 modulation significantly alters the AC consistently across the population.

Similar to AC, we also performed a cosine similarity measurement within and across the aCSF and D1 trials for STA curves, we performed a Kruskal--Wallis H test on three groups: within-aCSF, within-agonist, and across-aCSF and agonist conditions. The test revealed a significant effect of group on similarity distributions ($H(2) = 14.3199$, $p = 7.77e-4$).

Post hoc comparisons using Mann--Whitney U tests (Bonferroni-corrected for multiple comparisons) showed that:
\begin{itemize}
    \item Similarity scores were significantly higher within aCSF compared to across-pair comparisons ($U = 142428.00 $, $p = 7.44e-3$),
    \item D1 also showed significantly lower similarity than across-pair comparisons ($U = 134285.50$, $p = 1.95e-3$),
    \item We didn't observe a significant difference between aCSF and D1 mean cosine similarity distributions ($U = 142204.00$, $p = 1.0$).
\end{itemize}

These findings indicate that neural representations are more consistent within conditions than across conditions, suggesting that D1 modulation significantly alters the STA consistently across the population.

Finally, we assessed the effect of D1-R activation on action potential attributes (Fig. \textbf{\ref{fig:clustering} A.5}) which incorporates spiking dynamics, spiking threshold and AP height and width attributes for excitatory population. We found that max ISI (one sided t-test: t = 3.148
, p = 4.49e-3), mean ISI (one sided t-test: t = 2.283, p = 0.0319), instantaneous firing Rate (one sided t-test: t = -3.826, p = 8.64e-4) form the spiking dynamics are significantly altered as a result of D1-R activation. We performed similar analysis for D2 and M1 agonist trials and summarized the results in Fig. \ref{fig:S4} and Fig. \ref{fig:S5}.      

For inhibitory neurons, the effect on clustering in case of D1-R activation is shown in histogram (Fig. \textbf{\ref{fig:clustering} B.1}), it can be seen that AMI scores are consistently low for AP, PB, AC and STA, suggesting that functional attribute based clustering is altered as a result of D1R activation. We further explored how each attribute set is altered as a result of D1-R activation, for passive biophysical properties (Fig. \textbf{\ref{fig:clustering} B.2}) we found that conductance (gL) and reset voltage (Vr) are significantly reduced (One sided t-test: p<0.001, p<0.05) as a result of D1-R activation. We then explored how adaptation current and linear input filter (STA) are altered as a result of D1-R modulation shown in Fig. \textbf{\ref{fig:clustering} B.3} and Fig. \textbf{\ref{fig:clustering} B.4}. As for excitatory population, we calculated the rise time and peak values (see \ref{methods}) for both adaptation currents and STA. The joint plot Fig. \ref{fig:S6} shows the rise time and peak differences between D1 and aCSF trials for adaptation current and Fig. \ref{fig:S7} shows rise time and peak value for STA. The peak (paired t-test: t = -1.195, p = 0.244) and decay time (paired t-test: t = 0.7966, p = 0.4341) for adaptation current were not found to be significantly different. For STA, the rise time was found to be significantly different between aCSF and D1 trials (paired t-test: t = 2.454, p = 0.0208) but the peak current was not significantly different for inhibitory neurons (paired t-test: t = 1.834 p = 0.077). This suggest that STA and adaptation currents are not altered significantly as a result of D1-R activation. We also performed a cosine similarity measurement within and across the aCSF and D1 trials for AC curves, we performed a Kruskal--Wallis H test on three groups: within-aCSF, within-agonist, and across-aCSF and agonist conditions. The test revealed a non-significant effect between aCSF and D1 trials on similarity distributions ($H(2) = 2.0098$, $p = 0.3660$). These findings indicate that AC is not altered as a result of D1 modulation. Similarly, we performed a cosine similarity measurement within and across the aCSF and D1 trials for STA curves. The test revealed a significant effect of group on similarity distributions ($H(2) = 3.2486$, $p = 0.1970$). These findings indicate that AC is not altered as a result of D1 modulation. Finally, we assessed the effect of D1-R activation on action potential attributes (Fig. \textbf{\ref{fig:clustering} B.5}) which incorporates spiking dynamics, spiking threshold and AP height and width attributes for inhibitory population. We found that firing rate (one sided t-test: t = 3.627511899, p = 0.0011) from spiking dynamics subset are significantly altered, also mean threshold (one sided t-test: t = 4.1205, p = 3.038e-4), median threshold (one sided t-test: t = 4.361, p = 1.582e-4) and minimum threshold (one sided t-test: t = 2.443, p = 0.021) from the spiking threshold set were significantly altered as result of D1-R activation.    

We wanted to further understand if there are sub groups of neurons that are altered differently in their action potential and passive biophysical attributes as a result of D1R activation. For this, we clustered the change in AP and PB attributes between aCSF and D1 trials for both excitatory and inhibitory neurons and summarize our finding using polar plots with each set of attributes Fig. \textbf{\ref{fig:D1_diff}}, each curve represents a single neuron, colored with its respective cluster identity and the mean is represented with a thick line. It can be seen that there are 3 clusters of AP attributes for excitatory neurons and 4 clusters for inhibitory neurons. Similarly, we find 3 clusters each of PB attributes for both excitatory and inhibitory neurons. We also compared the overall similarity between clustering results based on aCSF trials and clustering the difference between aCSF and D1 trials for the 4 attribute sets using the AMI score between cluster labels. We summarized our findings in histogram shown in Fig. \ref{fig:D1_diff}. It can be seen that AMI score for both excitatory and inhibitory neurons consistently low for all 4 attributes for both excitatory and inhibitory neurons. We performed similar analysis for D2 and M1 agonist trials and summarized our findings in Fig. \textbf{\ref{fig:S9}} and Fig. \textbf{\ref{fig:S10}}.  

\subsection{Effect of neuromodualtion on information transfer capability in single neurons}

In order to assess the impact of neuromodulation on neuronal function, we first examined the relationship between firing rate changes and information transfer (fractional information, FI) across different recording conditions. Fig. \ref{fig:FI_change} \textbf{A} shows a scatter plot comparing the change in firing rate versus change in FI between two consecutive aCSF recordings, the values are normalized in order to be compared to agonist trials. We find that excitatory neurons mostly lie on the 3rd quadrant which means that firing rate and FI lowers as a result of consecutive aCSF trials. Inhibitory neurons on the other hand show much more heterogeneous response in change in firing rate and FI with mean centered around zero as seen from the size of the violin plots for both FI and firing rate between two aCSF trials suggesting heterogeneous trial to trial variability.   

We next investigated the effects of respective receptor activation on transferred information in a cell-type specific manner, as shown using box plot in Fig. \textbf{\ref{fig:FI_change} B}. We observed that D1-R activation resulted in a significant reduction in FI for excitatory neurons compared to inhibitory neurons (students t-test: t = -3.1667, p = 2.707e-3), underscoring a pronounced cell-type specific effect. In contrast, D2 receptor activation did not significantly alter FI in either neuronal population (students t-test: t = -1.5401, p = 0.1291). M1 receptor activation also produced a cell-type specific modulation, with excitatory neurons displaying a significant reduction in FI relative to inhibitory neurons (students t-test: t = -2.9521 1, p = 5.38e-3). Furthermore, the change in FI between consecutive aCSF trials was significantly lower in excitatory neurons than in inhibitory neurons (students t-test: t = -2.5738  1, p = 0.0118 ), reinforcing the idea that inhibitory neurons exhibit a broader trial to trial variability.

To further examine the effect of specific receptor activation, we plotted the normalized change in firing rate against FI for each neuromodulator compared to aCSF. The scatter plots in Fig. \ref{fig:FI_change} C (left panels) demonstrates that D1 activation alters the relationship between firing rate and FI, we can see that excitatory neurons show a much larger variance for change in firing rate and FI compared to inhibitory neurons. This suggest that excitatory neurons are modulated much more strongly than inhibitory neurons. The kernel density estimation (KDE) plots (right panels) compared the distribution of FI values between aCSF and D1 conditions. The KDE plot show a significant shift to the right in FI distribution under D1 activation for excitatory neurons, indicating that D1 receptor activation is significantly increases (Wilcoxon signed-rank test: stat = 110.0, p = 0.0190 ) the information transfer in excitatory neurons. Similarly, the scatter plots in Fig. \textbf{\ref{fig:FI_change}} \textbf{C-D} (left panels) demonstrates that D2-R and M1-R activation alters the relationship between firing rate and FI as well, we can see that excitatory neurons show a much larger variance for change in firing rate and FI compared to inhibitory neurons in case of D2. This suggest that excitatory neurons are modulated much more strongly than inhibitory neurons in case of D2-R activation. In case of M1-R activation, the difference in variance between excitatory and inhibitory neurons is much less pronounced. The kernel density estimation (KDE) plots (right panels) compared the distribution of FI values between aCSF and D1 conditions. For both M1-R (Wilcoxon signed-rank test: statistic = 22.0, p = 2.02e-3) and D2-R (Wilcoxon signed-rank test: stat = 225.0 , p = 0.01201) activation the FI distribution was significantly lowered for excitatory neurons, indicating that both M1 and D2 receptor activation significantly modulates the information transfer capabilities of excitatory neurons. 

Together, these results demonstrate that neuromodulation exerts cell-type specific effects on information transfer in neurons. D1 and M1 receptor activations significantly reduce FI in excitatory neurons compared to inhibitory neurons, while the effects of D2 activation are not significant. On the other hand, D1 receptor activation significantly increases information transfer for inhibitory neurons, while D2-R and M1-R significantly reduces information transfer for excitatory neurons. The differential changes observed in excitatory versus inhibitory populations, both under baseline conditions and following neuromodulatory interventions, suggest that neuromodulators play a critical role in redefining neuronal functional identity and information processing within neural circuits.

\section{Discussion}


In this study, we aimed to investigate how neuromodulation, specifically through dopamine (D1R, D2R) and acetylcholine (M1R) receptor activation, alters the functional properties of cortical neurons beyond traditional measures of excitability. Using frozen noise stimulation based single-cell in-vitro somatic recordings from layer 2/3 of the mouse somatosensory cortex, we extracted four functional feature sets comprising action potential (AP), passive biophysical (PB), adaptation currents (AC), and linear input filter via a spike-triggered average (STA). This approach enabled us to study the impact of neuromodulation across multiple physiological domains within and across neuronal subtypes.

\subsubsection{Receptor-Specific Neuromodulation Reshapes Functional Architecture in Distinct Cell Types}

Our analyses revealed that dopaminergic and cholinergic receptor activation altered the correlation structure among functional attributes, and these effects were excitatory/inhibitory cell-type specific. For inhibitory neuron\textbf{s}, D1R and D2R activation increased inter-attribute correlations, suggesting a convergence of functional attributes under neuromodulatory influence. In contrast, excitatory neurons displayed decreased correlations under D1R activation, indicating a decoupling of intrinsic and encoding features. This suggests that dopaminergic modulation sharpens functional coherence in inhibitory neurons while increasing functional independence in excitatory neurons. 

These results extend previous findings that D1R activation increases firing in both excitatory and inhibitory neurons in prefrontal and motor cortices~\cite{seamans2004principal, tritsch2012dopaminergic,anastasiades2019cell}, by showing that in the sensory cortex, such modulation also reorganizes functional coupling between key electrophysiological domains. Inhibitory neurons, under D1R and D2R activation, exhibited stronger coupling between passive properties, AP dynamics, and linear input filtering; suggesting that neuromodulation may reduce heterogeneity and impose a more unified computational role. This aligns with the hypothesis that dopamine can increase synchronization and gain control within inhibitory networks, potentially sharpening their influence on local circuits \cite{gao2003selective,morozova2016dopamine,seamans2001dopamine}.

Interestingly, we observed similar coupling effects under M1R activation, pointing to a convergent mechanism across dopaminergic and cholinergic systems in shaping the internal structure of inhibitory neuron function. The decoupling observed in excitatory neurons, particularly under D1R, may reflect a shift toward greater computational flexibility or specialization, allowing for more diverse input-output transformations. These effects may be a substrate for dynamic control of cortical processing modes, such as switching between attentive and exploratory states.

\subsubsection{Functional Clustering and Heterogeneous Modulation of Neuronal Identity}

To examine how these neuromodulatory changes reorganize functional neuron types, we applied UMAP-Louvain clustering to the high-dimensional feature sets before and after receptor activation. Our results show that agonist application drastically alters functional clustering, suggesting that receptor-specific modulation reshapes neuronal identities in both excitatory and inhibitory populations.

Importantly, the modulatory effects were not uniform, but heterogeneous within cell types, implying that neuromodulation acts in a subtype-specific manner. This is consistent with recent work showing volume transmission and receptor expression gradients as mechanisms for differential modulation across cell types~\cite{ozccete2024mechanisms}. By clustering neurons based on their change vectors (difference between agonist and control), we identified distinct subpopulations exhibiting coordinated shifts in AP and passive properties, reinforcing the idea that neuromodulation reconfigures the functional state space of neurons rather than simply scaling their excitability.

We also found that adaptation currents and STAs became more homogeneous in excitatory neurons compared to inhibitory neurons following neuromodulation. This suggests that although both populations undergo modulation, excitatory neurons may be pushed toward a more constrained encoding regime, possibly to provide stability in coding  under varying network conditions. Nonetheless, the core distinction between excitatory and inhibitory populations remained intact, highlighting the robustness of intrinsic identity despite substantial functional plasticity. We speculate that a modeling effort superimposing neuromodulatory affects on a balanced networks would reveal the implication of neuromodulatory alteration on the balance excitation-inhibition and therefore the shift that neuromodulation can cause.    

\subsubsection{Neuromodulation Selectively Alters Information Transfer}

While neuromodulators are known to affect neuronal excitability, less is known about how such changes impact information transmission. Using the frozen noise protocol~\cite{zeldenrust2017estimating}, we estimated fractional information (FI) between stimulus and spike train for each neuron under different receptor conditions.

Our key finding is that neuromodulation alters information transfer in a cell-type and receptor-specific manner. Specifically, D1R and M1R activation significantly decreased FI in excitatory neurons, while D2R activation increased FI in inhibitory neurons. This suggests that excitatory and inhibitory neurons are differentially engaged by neuromodulators to redistribute information processing roles. Moreover, FI variance increased in excitatory neurons under D1R, implying a diversification of encoding strategies, whereas inhibitory neurons exhibited more stable and coordinated changes.

These findings underscore an important result that modulation of biophysical and spike-generating properties has direct consequences on the information transfer and encoding of neurons. Neuromodulators do not simply shift the gain of neurons, they reconfigure how inputs are integrated and transformed into outputs, thus shaping the flow of information through cortical networks. This raises exciting new questions about how neuromodulatory systems shape perceptual inference, attention, and learning by dynamically allocating information processing across cell types.

\subsubsection{Limitations and Future Directions}

Our study, while comprehensive, has several important limitations. First, our recordings were limited to the soma, whereas neuromodulation often targets synaptic and dendritic compartments, which could substantially influence neuronal input-output transformations. Second, our analysis is constrained by sample size and cortical region, necessitating broader recordings across layers and brain areas to validate the generality of our findings.

Most critically, neuromodulatory systems are highly degenerate: the same ion channel can be targeted by multiple modulators, and a single neuromodulator can affect many channels and cellular processes. Thus, our receptor-specific results represent only a partial view of the underlying modulatory landscape. Future work should incorporate multi-receptor interactions, consider temporal dynamics of modulation, and examine how these single-cell changes propagate to network-level phenomena such as synchrony, attractor stability, and behavioral output.

\subsubsection{Conclusion}

Together, our findings reveal that D1, D2, and M1 receptor activation differentially reorganizes the biophysical, adaptive, and encoding properties of excitatory and inhibitory neurons in layer 2/3 of the somatosensory cortex. Neuromodulation can couple or decouple domains of neuronal function, alter functional classification, and modulate information transmission in a subtype-specific manner. These effects likely support context-dependent reconfiguration of cortical computation and offer a new window into how global modulatory signals dynamically orchestrate diverse local circuit functions.



\newpage
\begin{spacing}{1.0} % Set the line spacing to single spacing
\fontsize{8pt}{8pt}\selectfont
\bibliographystyle{apalike} %%%%Changed
\renewcommand{\bibname}{References}
\bibliography{All_bibtex} %%%%Changed
\end{spacing}